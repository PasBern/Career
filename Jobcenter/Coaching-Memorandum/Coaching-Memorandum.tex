\documentclass[11pt,a4paper]{article}
\usepackage{ngerman}
\usepackage[ngerman]{babel}
\usepackage[utf8x]{inputenc}
\usepackage[T1]{fontenc}
\usepackage{lmodern}
\usepackage{marvosym}
\usepackage{pifont}
\usepackage{ifpdf}
\usepackage[pdftex]{color}
\ifpdf
  \usepackage[pdftex]{graphicx}
\else
  \usepackage[dvips]{graphicx}\fi

\pagestyle{empty}

\usepackage[scale=0.775]{geometry}
\setlength{\parindent}{0pt}
\addtolength{\parskip}{6pt}

\def\firstname{Pascal}
\def\familyname{Bernhard}
\def\FileAuthor{\firstname~\familyname}
\def\FileTitle{\firstname~\familyname's Bewerbungsschreiben}
\def\FileSubject{Bewerbungsschreiben}
\def\FileKeyWords{\firstname~\familyname, Bewerbungsschreiben}

\renewcommand{\ttdefault}{pcr}
\hyphenation{ins-be-son-de-re}
\usepackage{url}
\urlstyle{tt}
\ifpdf
  \usepackage[pdftex,pdfborder=0,breaklinks,baseurl=http://,pdfpagemode=None,pdfstartview=XYZ,pdfstartpage=1]{hyperref}
  \hypersetup{
    pdfauthor   = \FileAuthor,%
    pdftitle    = \FileTitle,%
    pdfsubject  = \FileSubject,%
    pdfkeywords = \FileKeyWords,%
    pdfcreator  = \LaTeX,%
    pdfproducer = \LaTeX}
\else
  \usepackage[dvips]{hyperref}
\fi

\definecolor{firstnamecolor}{RGB}{56,115,179}
\definecolor{familynamecolor}{RGB}{56,115,179}
\hypersetup{pdfborder=0 0 0}

\begin{document}
\sffamily   % for use with a résumé using sans serif fonts;
%\rmfamily  % for use with a résumé using serif fonts;
\hfill%
\begin{minipage}[t]{.6\textwidth}
\raggedleft%
\includegraphics[width=0.55\textwidth]{Coaching-Logo_1280-720.jpg}


%	{\bfseries {\color{firstnamecolor}\firstname}~{\color{familynamecolor}\familyname}}\\[.35ex]
%	\small\itshape%
%	Schwalbacher Straße 7\\
%	12161 Berlin\\[.35ex]
%	\Mobilefone~+49 162 32 39 557 \\
%	\Letter~\href{mailto:pascal.bernhard@rppr.de}{pascal.bernhard@rppr.de}
\end{minipage}\\[0.5em]
%
{\color{firstnamecolor}\rule{\textwidth}{.25ex}}
%
\begin{minipage}[t]{.4\textwidth}
	\raggedright%
	% {\bfseries {\color{firstnamecolor}
	\vspace*{1em}
	\textbf{Jobcenter Berlin Tempelhof-Schöneberg} \\
	Frau Ecke \\[.35ex]
	% }}
	\small%
	Wolframstraße 89-92\\
	12105 Berlin
\end{minipage}
%
\hfill
%
\begin{minipage}[t]{.4\textwidth}
	\raggedleft % US style
	\today
	%April 6, 2006 % US informal style
	%05/04/2006 % UK formal style
\end{minipage}\\[0.2em]


{\bfseries \color{familynamecolor}{Weiterqualifizierung zum systemischen Coach -- Brückenschlag in den Arbeitsmarkt}}\\[0.75em]

Sehr geehrte Frau Ecke,\\[0.2em]
%
im Folgenden stelle ich Ihnen Kompetenzen und Aufgabenbereich des Berufsbilds \textsl{Coach} in der heutigen Dienstleistungsgesellschaft vor. Zugleich will ich die Inhalte von Coaching-Ausbildungen erläutern, welche Fähigkeiten dort vermittelt werden, wie eine Weiterqualifizierung zum Coach den eingeschlagenen Weg als strukturiert denkenden Politikwissenschaftler fortsetzt und meine Chancen als Berufseinsteiger erweitert und verbessert.

\subsection*{\textsf{Coach -- ein vielseitiges Berufsbild}}

\textsl{Coaching} ist eine sehr individualisierte Beratung, die es Menschen ermöglichen soll, ihre berufliche Rolle und Aufgaben bestmöglich zu erfüllen und hierbei auch persönlich zu fördern. Für die personal-strategische Zielsetzung von Unternehmen bedeutet dies, ihre Mitarbeiter mit Hilfe von internen oder externen Coaches zu weiterzuentwickeln, dass sie auch die an sie gestellten Anforderungen in Zukunft erfüllen können. Der Coach selbst soll Einzelne Mitarbeiter oder Teams zu bester Leistung führen ohne jedoch selbst an der Ausführung beteiligt zu sein, er gibt Impulse und zeigt neue Wege auf. 
\paragraph{\textsf{Anlässe für Coaching}}
\begin{itemize}
\item Veränderungen in der Personalstruktur eines Unternehmens

	\begin{itemize}
	\item Mitarbeiter werden neuen Abteilungen zugeordnet und müssen sich in neuen Teams zurechtfinden
	\item Durch Beförderung oder Arbeitsplatzwechsel steht die betreffende Personen vor\\
	\ding{225} neue Soft-Skills werden von den Mitarbeitern verlangt
	\end{itemize}

\item Fusionen und Übernahmen

	\begin{itemize}
	\item Unterschiedliche Unternehmenskulturen müssen zusammengeführt werden

		\begin{itemize}
		\item Zuständigkeiten und Aufgabengebiete in der neuen Unternehmensstruktur sind festzulegen\\
		\ding{225} neue Rollen des Personals müssen zuerst definiert werden\\
		\ding{225} Mitarbeiter stehen vor der Herausforderung, sich in ihren neuen Rollen zurechtzufinden
		\end{itemize}

\item Strategische Neuausrichtung eines Konzerns

	\begin{itemize}
	\item \textsl{Change Management} -- \textsl{Turn-Around} -- \textsl{Rationalisierung \& Sparmaßnahmen}
	\end{itemize}


	\end{itemize}
\end{itemize}



  
%Yours sincerely,\\[2em] % if the opening is "Dear Mr(s) Doe,"
%Mit freundlichen Grüßen,\\[2em] % if the opening is "Dear Sir or Madam,"
%
%\includegraphics[scale=0.75]{signature_blue}\\
%{\bfseries \firstname~\familyname}\\
%
%\vfill%
%{\slshape \bfseries Bewerbungsunterlagen}\\
% {\slshape Lebenslauf\\
% Arbeitszeugnisse\\
% Diplomzeugnis{}}
\end{document}
