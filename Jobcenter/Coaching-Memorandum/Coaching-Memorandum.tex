\documentclass[11pt,a4paper]{article}
\usepackage{ngerman}
\usepackage[ngerman]{babel}
\usepackage[utf8x]{inputenc}
\usepackage[T1]{fontenc}
\usepackage{lmodern}
\usepackage{marvosym}
\usepackage{amsfonts,amsmath,amssymb}
\usepackage{textcomp}
\usepackage{pifont}
\usepackage{ifpdf}
\usepackage[pdftex]{color}
\ifpdf
  \usepackage[pdftex]{graphicx}
\else
  \usepackage[dvips]{graphicx}\fi

\pagestyle{empty}

\usepackage[scale=0.775]{geometry}
\setlength{\parindent}{0pt}
\addtolength{\parskip}{6pt}

\def\firstname{Pascal}
\def\familyname{Bernhard}
\def\FileAuthor{\firstname~\familyname}
\def\FileTitle{\firstname~\familyname's Bewerbungsschreiben}
\def\FileSubject{Bewerbungsschreiben}
\def\FileKeyWords{\firstname~\familyname, Bewerbungsschreiben}

\renewcommand{\ttdefault}{pcr}
\hyphenation{ins-be-son-de-re}
\usepackage{url}
\urlstyle{tt}
\ifpdf
  \usepackage[pdftex,pdfborder=0,breaklinks,baseurl=http://,pdfpagemode=None,pdfstartview=XYZ,pdfstartpage=1]{hyperref}
  \hypersetup{
    pdfauthor   = \FileAuthor,%
    pdftitle    = \FileTitle,%
    pdfsubject  = \FileSubject,%
    pdfkeywords = \FileKeyWords,%
    pdfcreator  = \LaTeX,%
    pdfproducer = \LaTeX}
\else
  \usepackage[dvips]{hyperref}
\fi

\definecolor{firstnamecolor}{RGB}{56,115,179}
\definecolor{familynamecolor}{RGB}{56,115,179}
\hypersetup{pdfborder=0 0 0}

% Gleiche Schriftart für Hyperlinks
\urlstyle{same}


%  Gefrickel um URL-Links vernünftig umzubrechen
\makeatletter
\g@addto@macro\UrlBreaks{
  \do\a\do\b\do\c\do\d\do\e\do\f\do\g\do\h\do\i\do\j
  \do\k\do\l\do\m\do\n\do\o\do\p\do\q\do\r\do\s\do\t
  \do\u\do\v\do\w\do\x\do\y\do\z\do\&\do\1\do\2\do\3
  \do\4\do\5\do\6\do\7\do\8\do\9\do\0}
% \def\do@url@hyp{\do\-}

% Hiermit soll einer übervolle Box verhindert werden -- funktioniert sogar irgendwie
\g@addto@macro\UrlSpecials{\do\/{\mbox{\UrlFont/}\hskip 0pt plus 1pt}}
\makeatother

% Farben werden hier definiert
\definecolor{MidnightBlue}{RGB}{0,103,149}


\begin{document}
\sffamily   % for use with a résumé using sans serif fonts;
%\rmfamily  % for use with a résumé using serif fonts;
\hfill%
\begin{minipage}[t]{.6\textwidth}
\raggedleft%
\includegraphics[width=0.55\textwidth]{Coaching-Logo_1280-720.jpg}


%	{\bfseries {\color{firstnamecolor}\firstname}~{\color{familynamecolor}\familyname}}\\[.35ex]
%	\small\itshape%
%	Schwalbacher Straße 7\\
%	12161 Berlin\\[.35ex]
%	\Mobilefone~+49 162 32 39 557 \\
%	\Letter~\href{mailto:pascal.bernhard@rppr.de}{pascal.bernhard@rppr.de}
\end{minipage}\\[0.5em]
%
{\color{firstnamecolor}\rule{\textwidth}{.25ex}}
%
\begin{minipage}[t]{.4\textwidth}
	\raggedright%
	% {\bfseries {\color{firstnamecolor}
	\vspace*{1em}
	\textbf{Jobcenter Berlin Tempelhof-Schöneberg} \\
	Frau Ecke \\[.35ex]
	% }}
	\small%
	Wolframstraße 89-92\\
	12105 Berlin
\end{minipage}
%
\hfill
%
\begin{minipage}[t]{.4\textwidth}
	\raggedleft % US style
	\today
	%April 6, 2006 % US informal style
	%05/04/2006 % UK formal style
\end{minipage}\\[2.2em]


{\bfseries \color{familynamecolor}{Weiterqualifizierung zum Systemischen Coach -- Brückenschlag in den Arbeitsmarkt}}\\[0.75em]

Sehr geehrte Frau Ecke,\\[0.2em]
%
im Folgenden stelle ich Ihnen Kompetenzen und Aufgabenbereiche des Berufsbilds \textsl{Coach} in der heutigen Dienstleistungsgesellschaft vor. Zugleich will ich die Inhalte von Coaching-Ausbildungen erläutern, welche Fähigkeiten dort vermittelt werden. So wird ersichtlich wie eine Weiterqualifizierung zum Coach den eingeschlagenen Weg als strukturiert denkenden Politikwissenschaftler fortsetzt und meine Chancen als Berufseinsteiger erweitert und verbessert.

\subsection*{\textsf{Coach -- ein vielseitiges Berufsbild}}


\textsl{Coaching} ist eine sehr individualisierte Beratung, die es Menschen ermöglichen soll, ihre (berufliche) Rolle und Aufgaben bestmöglich zu erfüllen und die Klienten auch persönlich fördert. Der Coach soll einzelne Mitarbeiter oder Teams zu bester Leistung führen ohne jedoch selbst an der Ausführung beteiligt zu sein -- er gibt Impulse und zeigt neue Wege auf. Es existieren viele unterschiedliche Coaching-Arten, die vom Leistungssport, dem Ursprung des modernen Coachings, über das Personal-Coaching zur Erreichung persönlicher Lebensziele zum Führungskräfte-Coaching reichen.

Als Politikwissenschaftler, der gelernt hat in Strukturen zu denken und komplexe Zusammenhänge zu erfassen, strebe ich eine Karriere auf dem Feld des systemischen Coachings zur Organisationsentwicklung an. Im Unterschied zu Personal-Coaches bestehen im systemischen Coaching die Möglichkeiten als interne Berater in Festanstellung für Unternehmen zu arbeiten. Eine Karriere auf freiberuflicher Basis als externer Coach, welcher von Firmen für konkrete Coaching-Anlässe engagiert wird, steht ebenfalls offen. Für die personal-strategische Zielsetzung von Unternehmen haben Coaches die Aufgabe, Mitarbeiter zu weiterzuentwickeln, sodass sie auch die an sie gestellten Anforderungen in Zukunft erfüllen können. Coaching-Kompetenzen gewinnt auch in anderen beruflichen Rollen an Bedeutung bzw. verändert diese Rollen innerhalb von Unternehmen und Organisationen: Personalentwickler, Projektleiter und Unternehmensberater profitieren ebenfalls von Coaching-Kompetenzen in der Teamarbeit und Mitarbeiterführung.

\newpage

\paragraph{\textsf{Anlässe für Coaching -- Wann sind Coaches gefordert?}}
\begin{itemize}
\item \textbf{Veränderungen in der Personalstruktur eines Unternehmens}

	\begin{itemize}
	\item Mitarbeiter werden neuen Abteilungen zugeordnet und müssen sich in neuen Teams zurechtfinden
	\item Durch Beförderung oder Arbeitsplatzwechsel steht die betreffenden Personen vor neuen Aufgaben\\
	\ding{225} neue Soft-Skills werden von den Mitarbeitern verlangt
	\end{itemize}

\item \textbf{Fusionen und Übernahmen}

	\begin{itemize}
	\item Unterschiedliche Unternehmenskulturen müssen zusammengeführt werden

		\begin{itemize}
		\item[\textbullet] Zuständigkeiten und Aufgabengebiete in der neuen Unternehmensstruktur sind festzulegen\\
		\ding{225} Neue Rollen des Personals müssen definiert werden\\
		\ding{225} Mitarbeiter stehen vor der Herausforderung, sich in ihren neuen Rollen zurechtzufinden
		\end{itemize}

	\item Strategische Neuausrichtung eines Konzerns

		\begin{itemize}
		\item[\textbullet] \textsl{Change Management} -- \textsl{Turn-Around} -- \textsl{Rationalisierung \& Sparmaßnahmen}\\
		\ding{225} Anforderungen an Mitarbeiter verändern sich grundlegend\\
		\ding{225} Das Personal muss lernen, mit reduzierten Ressourcen (Arbeitszeit / Budget) umzugehen\\
		\ding{225} Kompetenzen für effizienteres Arbeiten sind gefragt\\
		\ding{225} Neue Problemlösungsstrategien müssen erlernt werden\\
		\ding{225} Kritische Selbstreflexion zu eigener Arbeitsweise, Gruppendynamiken, Feed-Back--Mechanismen, Führungsstil
		\end{itemize}


	\end{itemize}
\end{itemize}

\subsection*{\textsf{Ausbildung zum Coach}}

Da der Begriff \textsl{Coach} nicht rechtlich geschützt ist und es hierbei nicht um einen klassischen Ausbildungsberuf handelt, kann sich jede und jeder selbst als 'Coach' bezeichnen. So hat eine vom Deutschen Bundesverband Coaching e.V. (DVBC) zertifizierte Ausbildung und die anschließende Anerkennung als qualifizierter Coach einen hohen Stellenwert in der Branche. Für den Berufseinstieg nach der Ausbildung sind die Mitgliedschaft im Branchenverband\footnote{Deutscher Bundesverband Coaching e.V.\\ \textsf{\textcolor{MidnightBlue}{\url{http://www.dbvc.de/aufnahme-in-den-dbvc/mitgliedschaftsformen.html}}}} und die Aufnahme in Deutschlands bedeutendste Coaching-Datenbank, die \textsl{Rauen Coach-Datenbank}, entscheidend\footnote{Rauen Datenbank:\\\textsf{\textcolor{MidnightBlue}{http://www.coach-datenbank.de/aufnahme\_in\_die\_coach-datenbank.htm}}}.

\subsubsection*{\textsf{Inhalte der Coaching-Ausbildung}}

Eine Coaching-Ausbildung baut auf bestehenden fachspezifischen Kompetenzen auf, die im Zuge der Weiterqualifizierung mit neuen Fähigkeiten und Wissen verknüpft werden, sodass Coaches ein umfassendes Repertoire aus persönlichen und fachlichen Kompetenzen anbieten können. Grundlegende Voraussetzung für eine Ausbildung ist ein erfolgreich abgeschlossenes Studium der Sozialwissenschaften: die meisten Coaches kommen aus den Bereichen Betriebswirtschaft, Politikwissenschaft und Psychologie. Das zuvor im Studium erworbene fachliche Wissen und entsprechende Methoden zur Problemanalyse und -lösung bieten die Grundlage mit einer Qualifikation als Coach in dieser Branche tätig zu sein, da Aufgaben und Inhalte der Arbeit der gecoachten Personen vertraut sind.

\paragraph*{\textsf{Ziele einer Ausbildung zum systemischen Coach}}


\begin{itemize}

\item Theorien zu Lernen und Veränderung auf Grundlage von Konstruktivismus, Systemtheorie, Gestaltpsychologie, Mediation und Humanistischer Psychologie

	\begin{itemize}
		\item Vermittlung von Interventionsmethoden, um Klienten bei persönlicher Veränderung zu unterstützen
		\item erlernte Interventionstechniken mit Analysefähigkeiten verknüpfen, welche im Studium erworben wurden, um Unternehmen bei organisationellen Veränderungsprozessen zu beraten
	\end{itemize}


\item Erlernen unterschiedlicher Coachingansätze und insbesondere die Entwicklung eines persönlichen Coaching-Stils basierend auf individuellen fachlichen Kompetenzen\\
\ding{225} gezielte Profilierung als systemischer Coach

\item Projektmanagement im Kontext von Mitarbeiterführung und \textsl{Change Management}

\item Aufbau eines Netzes an Kontakten während der Coaching-Ausbildung

\item Unterstützung für die Aufnahme als Mitglied im Deutschen Bundesverband Coaching e.V. (DBVC)


\paragraph*{\textsf{Neue Kompetenzen durch eine Coaching-Ausbildung}}

\begin{enumerate}
\item Fähigkeit, emotionale und kognitive Selbstorganisation anzuregen
\item sog. \textsl{integriertes Organisationswissen} entwickeln: Kenntnisse aus dem Politikstudium zu komplexen Entscheidungsprozessen mit Inhalten der Ausbildung zu Gruppendynamiken, Psychologie und Lerntechniken verknüpfen und auf einen organisationellen Kontext anwenden
\item über die Analyse der Ziele und Anforderungen von Unternehmen hinausgehend diese in der Arbeitsweise der Mitarbeiter zu verankern\\
\ding{225} Hilfe zur Selbsthilfe auf professionellem Niveau anbieten zu können
\item während Ausbildung durch branchenspezifische Interventionsübungen und 'Case-Studies' Erfahrung in Beratung und Organisationsentwicklung gewinnen
\item Erwerb der Kompetenz professioneller Auftragsabwicklung
\end{enumerate}

\end{itemize}

\newpage

\subsection*{\textsf{Ausbildungen zu systemischem Coaching}}


% Artop-Ausbildung

\paragraph{\textsf{artop GmbH}} Institut an der Humboldt Universität zu Berlin: \textsl{Ausbildung zum Coach}\\
Die Coaching-Ausbildung wendet sich an einen Personenkreis aus Personalentwicklern, Unternehmensberatern, Führungskräften und Quereinsteigern. Absolventen der artop-Ausbildung zum Coach erhalten das Zertifikat „Systemischer Coach“ nach Teilnahme an den Seminareinheiten (mindestens 80\% Anwesenheit), dem Nachweis zweier eigenständig absolvierter Coachingprozesse, des Lehrcoachings und der Fallpräsentation auf dem Abschlusskolloquium. Die Ausbildung ist zertifiziert vom Deutschen Bundesverband Coaching e.V. Der Ausbildungsgang wird von der Senatsverwaltung für Arbeit, Integration und Frauen Berlin gemäß des Berliner Bildungsurlaubsgesetzes als Bildungsveranstaltung anerkannt.\\
\textbf{Webseite:}\textsf{\textcolor{MidnightBlue}{\url{http://www.artop.de/ausbildung-zum-coach-informationen}}}

Die Ausbildung umfasst vier Potenzialgespräche zu Beginn, 12 Ausbildungswochenenden, Lernfortschrittsgespräche, Supervision, zwei eigene Coaching-Prozesse, sowie 5 Einheiten von Lerncoaching (hierfür fallen gesonderte Kosten an). Die Qualifizierung mit einem Gesamtumfang von 210 Einheiten \`{a} 45 Minuten findet in Berlin statt.
	\begin{itemize}
	\item \textbf{Inhalte der Ausbildung:}
		
		\begin{enumerate}
		\item Grundlagen des Coachings
		\item Systemtheorie und systemisches Coaching
		\item Auftragsklärung im Coaching
		\item Testverfahren im Coaching
		\item Gruppendynamische Selbsterfahrung
		\item Einzelcoaching I
		\item Einzelcoaching II
		\item Gestaltmethoden
		\item Aufstellungsarbeit im Coaching
		\item Konfliktcoaching
		\item Mein Profil als Coach
		\item Coaching im Kontext von Teamentwicklung und Organisationsberatung
		\item Abschlusskolloquium
		\end{enumerate}
	
	\item \textbf{Dauer:}

		\begin{itemize} 
		\item voraussichtlich (Termin stehen laut Webseite noch nicht fest): September 2015 bis März 2017
		\end{itemize}

	\item \textbf{Kosten der Ausbildung:} für Privatpersonen 7.900 Euro /für Unternehmen 9.500 Euro

	\item \textbf{Bewerbungsverfahren:} Die Auswahl der Teilnehmer erfolgt über eine schriftliche Bewerbung mittels tabellarischem Lebenslauf, der ein sog. 'Potenzialgespräch' zu Eignung und Lernziele der Ausbildung folgt.
	\end{itemize}


% ISCO-AG Ausbildung

\paragraph{\textsf{ISCO AG}} Institut für Systemisches Coaching und Organisationsberatung AG: \textsl{Weiterbildung zum Systemischen Coach}\\
Die Weiterbildung zum Systemischen Coach richtet sich hauptsächlich an Trainer, Supervisoren, Führungskräfte, Unternehmensberater und Personen, die bereits als Coach tätig sind. Nach erfolgreichem Abschluss der Module erhalten die Teilnehmer das Zertifikat 'Systemischer Coach'. Die Ausbildung ist vom Deutschen Bundesverband Coaching e.V. akkreditiert.\\

\textbf{Webseite:}\textsf{\textcolor{MidnightBlue}{\url{http://isco-ag.de/coaching-weiterbildung/systemischer-coach-berlin.html}}}

Die Ausbildung findet in 7 Modulen (2-4 Tage) in Berlin oder Frankfurt am Main statt. Die Ausbildungsblöcke werden flankiert von weiteren 10 Tagen mit Live-Coaching, Reflexion in Gruppenarbeit und Supervision.

	\begin{itemize}
	\item \textbf{Inhalte der Ausbildung:}

			\begin{enumerate}
			\item Grundlagen des Coachings
			\item Potenzialanalyse
			\item Kommunikationsfähigkeit und Sprachgebrauch im Coaching
			\item Motivieren und Führen von Teams
			\item Supervision der Live-Coachings
			\item Selbsterfahrung
			\item Lösungsorientierte Coaching-Techniken aus dem NLP (Neuro-Linguistisches Programmieren\footnote{Erläuterung von NLP: \textsf{\textcolor{MidnightBlue}{\url{http://www.dvnlp.de/nlp-methode.html}}}})
			\item Rang und seine Bedeutung im Coaching
			\item Profilschärfung für Coaches
			\item Lernerfolgskontrolle
			\end{enumerate}

	\item \textbf{Dauer:}

		\begin{itemize} 
		\item \textsl{Berlin}: Juni 2015 bis circa August 2016
		\item \textsl{Frankfurt}: Februar 2016 bis circa März 2017
		\end{itemize}

	\item \textbf{Kosten der Ausbildung:} für Privatpersonen 6.535 Euro (zzgl. Mehrwertsteuer) /für Unternehmen 8.750 Euro (zzgl. Mehrwertsteuer)\\

	\item \textbf{Bewerbungsverfahren:} Die Auswahl der Teilnehmer erfolgt über ein erstes Kennenlern- und Eignungserhebungsgespräch. Es fällt eine Bearbeitungsgebühr von 100 Euro bis zum Abschluss des Ausbildungsvertrages an.

	\end{itemize}

% CA-Coaching Akademie-Ausbildung


\paragraph{\textsf{CA Coaching Akademie}} \textsl{Weiterbildung zum Coach der Wirtschaft (IHK)}\\
Die Qualifizierung zum Coach der Wirtschaft ist für Berater, Trainer, Personaĺ- und Organisationsentwickler, sowie Führungskräfte aus Wirtschaft und Verwaltung gedacht. Voraussetzung für die Weiterbildung ist ein erfolgreich abgeschlossenes Hochschulstudium oder ein entsprechenden beruflichen Werdegang. Nach gelungenem Abschluss der Ausbildung wird das Zertifikat "`Coach der Wirtschaft"' verliehen. Es findet eine Prüfung bestehend aus Präsentation eines individuellen Coaching-Konzeptes, mündlicher Prüfung zu Lerninhalten und Simulation einer Coaching-Situation statt. Die Ausbildung ist vom Deutschen Bundesverband Coaching e.V. anerkannt.\\

\textbf{Webseite:}\textsf{\textcolor{MidnightBlue}{\url{http://www.coaching-akademie.de/}}}


Die Ausbildung hat einen Gesamtumfang von 290 Stunden.


	\begin{itemize}
	\item \textbf{Inhalte der Ausbildung:}

			\begin{enumerate}
			\item Grundlagen des Coachings
			\item Vermittlung von unterschiedlichen Coaching-Ansätzen auf breiter theoretischen Basis
			\item Reflektierte Selbststeuerung in Coaching-Prozessen
			\item Übung zu werte- und zielorienter Steuerung in der Arbeitswelt
			\item Auswahl und Einsatz passender Coaching-Methoden
			\item Problemdistanz zur Professionalisierung im Coaching 
			\item Erlernen der praxisnahen Anwendung von Coaching-Ansätzen
			\item Erarbeiten eines individuellen Coaching-Konzeptes als Basis und Starthilfe zum Berufseinstieg
			\end{enumerate}

	\item \textbf{Dauer:}

		\begin{itemize} 
		\item \textsl{Hannover}: voraussichtlich Oktober 2015 bis Oktober 2016
		\item \textsl{Berlin/Potsdam}: voraussichtlich Oktober 2015 bis Oktober 2016
		\end{itemize}

	\item \textbf{Kosten der Ausbildung:} für Privatpersonen 5.950 Euro (zzgl. Mehrwertsteuer) /für Unternehmen 6.950 Euro (zzgl. Mehrwertsteuer). Des Weiteren fällt eine Prüfungsgebühr in Höhe von 450 Euro an, sowie eine Tagungspauschale mit 660 Euro für Hannover (ohne Übernachtung und Verpflegung), bzw. 950 Euro für Potsdam (ohne Übernachtung und Verpflegung)

	\item \textbf{Bewerbungsverfahren:} In einem Aufnahmegespräch werden Voraussetzungen der Bewerber und persönliche Zielvorstellungen geklärt.


	\end{itemize}

Je nach Ausbildungsstandort fallen zusätzliche Kosten für Anreise und Unterkunft an.

\newpage

\subsection*{\textsf{Aktuelle Stellenangebote im Bereich Coaching \& Change Management}}


\begin{itemize}

\item \textbf{Managing Change Consultant} -- \textsl{RWE Consulting GmbH}:\\ \textsf{\textcolor{MidnightBlue}{\url{http://www.stepstone.de/stellenangebote--MANAGING-CHANGE-CONSULTANT-M-F-Essen-Berlin-Frankfurt-Muenchen-RWE-Consulting-GmbH--3320829-inline.html?cid=JaJob-japu-homepage-05-2015_ps_1_4_offertitle&jacid=8385193-05-2015&nctid=20150528&intcid=JAtest_APPDOWNLOAD_A&bl=m}}}

\item \textbf{Personalleiter} -- \textsl{Christoffel-Blindenmission}:\\
\textsf{\textcolor{MidnightBlue}{\url{https://www.cbm.de/ueber-uns/stellenangebote/Personalleiter-w-m-483577.html}}}

\item \textbf{Trainer/Coach} -- \textsl{DHL}:\\
\textsf{\textcolor{MidnightBlue}{\url{http://stellenanzeige.monster.de/trainer-coach-m-w-bonn-job-bonn-nordrhein-westfalen-deutschland-147542937.aspx}}}

\item \textbf{Referent Personalentwicklung} -- \textsl{dfv Mediengruppe}:\\
\textsf{\textcolor{MidnightBlue}{\url{http://www.dfv.de/karriere}}}

\item \textbf{Mitarbeiter Personalentwicklung} -- \textsl{ALDI Süd}:\\
\textsf{\textcolor{MidnightBlue}{\url{http://www.stepstone.de/stellenangebote--Mitarbeiter-Personalentwicklung-m-w-Bous-spaeter-Muelheim-an-der-Ruhr-ALDI-SUeD--3328705-inline.html?isHJ=false\&isHJR=false&ssaPOP=3\&ssaPOR=3}}}

\item \textbf{Junior Referent Führungskräfteentwicklung} -- \textsl{Deloitte Consulting}:\\
\textsf{\textcolor{MidnightBlue}{\url{https://careers.deloitte.com/jobs/eng-global/details/j/E15-ES-AD-PE-254/junior-referent-m/w-f\%C3\%BChrungskr\%C3\%A4fteentwicklung?src=JB-16801\sharp}}}

\item \textbf{Beratungsassistenz / Junior Consultant Human Capital Analytics} -- \textsl{Kienbaum}:\\
\textsf{\textcolor{MidnightBlue}{\url{http://www.kienbaum.de/desktopdefault.aspx/tabid-188/320_read-1844/}}}

\item \textbf{Trainer / Performance Coach} -- \textsl{Austin Fraser}:\\
\textsf{\textcolor{MidnightBlue}{\url{http://stellenanzeige.monster.de/trainer-performance-coach-job-munich-bayern-deutschland-151122241.aspx}}}

\end{itemize}

\vspace{2cm}

Angesichts der Energiewende, sehen sich insbesondere Deutschlands große Energieversorger wie auch kleinere Stadtwerke vor der Herausforderung, ihre Geschäftsmodelle und entsprechend ihre Konzernstrukturen dem neuen regulativen Umfeld anzupassen. Der Wandel der Branche wird in erster Linie an die Mitarbeiter der Energieunternehmen neue Anforderungen stellen. Hier ist in den kommende Jahren ein deutlich verstärkte Beratungs- und Coachingbedarf zu erwarten. Hier einige Links mit weiterführenden Informationen zum Wandel in der Energiebranche:

	\begin{itemize}
	\item Herausforderungen von Stadtwerken aus der Energiewende (\textsl{Universität Leipzig}):\\
	\textsf{\textcolor{MidnightBlue}{\url{http://www.kompetenzzentrum-uni-leipzig.de/herausforderungen-von-stadtwerken-aus-der-energiewende/}}}

	\item Energieversorger: Marktliberalisierung, Energiewende und ihre Preismodelle (\textsl{Merit Commodity Management}):\\
	\textsf{\textcolor{MidnightBlue}{\url{http://www.meritcm.com/branchen/energie.html}}}

	\item Kulturwandel in der Energiebranche -- Herausforderungen für das Personal- und Organisationsmanagement (\textsl{The Advisory House}):\\
	\textsf{\textcolor{MidnightBlue}{\url{http://www.advisoryhouse.com/UserData/Publication_1370521098.pdf}}}
	\end{itemize}
	
Als Politikwissenschafter, der sich im Studium auf Energiepolitik spezialisiert hat, sehe ich in dieser Branche ein breitgefächertes Arbeitsplatzpotential, wo ich meine bisherigen Kenntnisse mit neuen Kompetenzen einer Coaching-Ausbildung verbinden kann. Unter potentiellen Arbeitgebern, die Mitarbeiter mit Coaching-Qualifikationen suchen, sei folgend eine kleine Auswahl genannt:

	\begin{itemize}
	\item Tiba Management Beratung GmbH\\
	\textsf{\textcolor{MidnightBlue}{\url{https://www.tiba.de/}}} 

	\item Morgenweck \& Company Business Consultants\\
	\textsf{\textcolor{MidnightBlue}{\url{http://www.morgenweck-company.com/}}}

	\item Dr. Kraus \& Partner\\
	\textsf{\textcolor{MidnightBlue}{\url{http://www.kraus-und-partner.de/}}}

	\item Bluebridges\\
	\textsf{\textcolor{MidnightBlue}{\url{http://www.bluebridges.de/}}}

	\item Becker Büttner Held Consulting AG\\
	\textsf{\textcolor{MidnightBlue}{\url{http://www.bbh-beratung.de}}}

	\end{itemize}


\end{document}