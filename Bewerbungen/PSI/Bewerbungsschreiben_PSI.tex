\documentclass[11pt,a4paper]{article}
\usepackage{ngerman}
\usepackage[ngerman]{babel}
\usepackage[utf8x]{inputenc}
\usepackage[T1]{fontenc}
\usepackage{lmodern}
\usepackage{marvosym}
\usepackage{ifpdf}
\usepackage[pdftex]{color}
\ifpdf
  \usepackage[pdftex]{graphicx}
\else
  \usepackage[dvips]{graphicx}\fi

\pagestyle{empty}

\usepackage[scale=0.775]{geometry}
\setlength{\parindent}{0pt}
\addtolength{\parskip}{6pt}

\def\firstname{Pascal}
\def\familyname{Bernhard}
\def\FileAuthor{\firstname~\familyname}
\def\FileTitle{\firstname~\familyname's Bewerbungsschreiben}
\def\FileSubject{Bewerbungsschreiben}
\def\FileKeyWords{\firstname~\familyname, Bewerbungsschreiben}

\renewcommand{\ttdefault}{pcr}
\hyphenation{ins-be-son-de-re}
\usepackage{url}
\urlstyle{tt}
\ifpdf
  \usepackage[pdftex,pdfborder=0,breaklinks,baseurl=http://,pdfpagemode=None,pdfstartview=XYZ,pdfstartpage=1]{hyperref}
  \hypersetup{
    pdfauthor   = \FileAuthor,%
    pdftitle    = \FileTitle,%
    pdfsubject  = \FileSubject,%
    pdfkeywords = \FileKeyWords,%
    pdfcreator  = \LaTeX,%
    pdfproducer = \LaTeX}
\else
  \usepackage[dvips]{hyperref}
\fi

\definecolor{firstnamecolor}{RGB}{125,85,85}
\definecolor{familynamecolor}{RGB}{138,74,57}
\hypersetup{pdfborder=0 0 0}

\begin{document}
\sffamily   % for use with a résumé using sans serif fonts;
%\rmfamily  % for use with a résumé using serif fonts;
\hfill%
\begin{minipage}[t]{.6\textwidth}
	\raggedleft%
	{\bfseries {\color{firstnamecolor}\firstname}~{\color{familynamecolor}\familyname}}\\[.35ex]
	\small\itshape%
	Schwalbacher Straße 7\\
	12161 Berlin\\[.35ex]
	\Mobilefone~+49 162 32 39 557 \\
	\Letter~\href{mailto:pbernhard@seitenfreunde.de}{pbernhard@seitenfreunde.de}
\end{minipage}\\[0.5em]
%
{\color{firstnamecolor}\rule{\textwidth}{.25ex}}
%
\begin{minipage}[t]{.4\textwidth}
	\raggedright%
	% {\bfseries {\color{firstnamecolor}
	\vspace*{1em}
	PSI AG \\
	Eve Cordes-Gassan\\[.35ex]
	% }}
	\small%
	Dircksenstraße 42-44\\
	10178 Berlin
\end{minipage}
%
\hfill
%
\begin{minipage}[t]{.4\textwidth}
	\raggedleft % US style
	\today
	%April 6, 2006 % US informal style
	%05/04/2006 % UK formal style
\end{minipage}\\[1em]
\raggedright

{\bfseries \color{familynamecolor}Bewerbung als Berater}\\[1.5em]

Sehr geehrte Frau Cordes-Gassan,\\[1em]
%
welche Folgen ergeben sich für Energieunternehmen, wenn Marktliberalisierung gesellschaftlich unerwünschte Effekte zeigt und Politiker Wählerstimmen gewinnen wollen? Wieso folgt Energiepolitik nicht wirtschaftlicher Logik oder den Erfordernissen des Klimaschutzes? Der Ukraine-Russland Konflikt stellt die Versorgungssicherheit mit Erdgas in Frage, aber weiterhin fehlt es an einer gemeinsamen europäische Energiepolitik? Einfache Antworten auf diese Fragen habe ich ganz offen gesagt auch nicht. Aber als Politikwissenschaftler kann ich die komplexen Zusammenhänge zwischen Energiemärkten und politischen Entscheidungen durchdringen und Lösungskonzepte für Unternehmen erarbeiten.


Mein Politikstudium habe ich in Berlin und Paris absolviert. Entscheidungsprozesse in Wirtschaft und Politik zu analysieren, waren Kernstück des Studiums an der Freien Universität. In vier sehr praxis-orientierten Semestern an Sciences Po Paris im MBA-Studiengang "Finance et Stratégie" habe ich  auch unternehmerische Denkweisen und Lösungsansätze kennengelernt. Verhandlungssichere Französisch- und Englischkenntnisse kann ich in Ihrem Berater-Team dank mehrjähriger Auslandsaufenthalte in Frankreich und den USA einbringen.


An der AmCham in Paris wie auch bei der Unternehmensberatung SCI Verkehr waren Recherchen für Marktstudien und Wettbewerberanalysen wesentlicher Bestandteil meiner Arbeit. Politische Rahmenbedingungen für Unternehmen zuverlässig einzuschätzen, zählte zu meinen Hauptaufgaben. Unsere Kunden waren über präzise Marktanalysen hinausgehend an Lösungen interessiert, wie sie ihre Geschäftsstrategie auf regulative Risiken und Politikänderungen ausrichten können. Hierbei konnte ich Kenntnisse über die unterschiedlichen Branchen Utilities und Schienenverkehr gewinnen, die ebenso wie die Energiewirtschaft stark von politischen Entscheidungen beeinflusst sind. 


Meine gegenwärtige Tätigkeit als Coach für Studenten vereint die Kompetenzen, einerseits komplexe Sachverhalte schnell zu durchdringen und zugleich Lösungen für Schwierigkeiten bei wissenschaftlichen Arbeiten zu entwickeln. Diese Konzepte verständlich zu kommunizieren, ist zweites Kernstück meiner Coaching-Arbeit. Zu meinen besonderen Stärken gehören eine schnelle Auffassungsgabe und analytisches Denkvermögen, die auch unter Zeitdruck nicht leiden.


Sie wünschen sich Mitarbeiter, die komplexe Zusammenhänge verstehen können, sich nicht vor neuen Aufgabengebieten scheuen und in der Lage sind, ihre Ergebnisse verständlich an Nicht-Experten zu vermitteln? Hier haben sie jemanden, der Ihnen diese Fähigkeiten anbietet.

Gerne überzeuge ich Sie in einem persönlichen Gespräch, dass sie in mir einen ebenso erfahrenen wie engagierten Mitarbeiter gewinnen.

Aufgrund meiner Qualifikation und Kenntnisse liegen meine Gehaltsvorstellungen bei 45.000 Euro im Jahr.


  
%Yours sincerely,\\[2em] % if the opening is "Dear Mr(s) Doe,"
Mit freundlichen Grüßen,\\[3em] % if the opening is "Dear Sir or Madam,"
%
%\includegraphics[scale=0.75]{signature_blue}\\
{\bfseries \firstname~\familyname}\\
%
\vfill%
{\slshape \bfseries Bewerbungsunterlagen}\\
 {\slshape Lebenslauf\\
 Arbeitszeugnisse\\
 Diplomzeugnis{}}
\end{document}
