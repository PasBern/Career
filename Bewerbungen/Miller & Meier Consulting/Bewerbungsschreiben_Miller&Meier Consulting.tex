\documentclass[11pt,a4paper]{article}
\usepackage{ngerman}
\usepackage[ngerman]{babel}
\usepackage[utf8x]{inputenc}
\usepackage[T1]{fontenc}
\usepackage{lmodern}
\usepackage{marvosym}
\usepackage{amsfonts,amsmath,amssymb}
\usepackage{textcomp}
\usepackage{ifpdf}
\usepackage{pifont}
\usepackage[pdftex]{color}
\ifpdf
  \usepackage[pdftex]{graphicx}
\else
  \usepackage[dvips]{graphicx}\fi

\pagestyle{empty}

\usepackage[scale=0.775]{geometry}
\setlength{\parindent}{0pt}
\addtolength{\parskip}{6pt}

\def\firstname{Pascal}
\def\familyname{Bernhard}
\def\FileAuthor{\firstname~\familyname}
\def\FileTitle{\firstname~\familyname's Bewerbungsschreiben}
\def\FileSubject{Bewerbungsschreiben}
\def\FileKeyWords{\firstname~\familyname, Bewerbungsschreiben}

\renewcommand{\ttdefault}{pcr}
\hyphenation{ins-be-son-de-re}
\usepackage{url}
\urlstyle{tt}
\ifpdf
  \usepackage[pdftex,pdfborder=0,breaklinks,baseurl=http://,pdfpagemode=None,pdfstartview=XYZ,pdfstartpage=1]{hyperref}
  \hypersetup{
    pdfauthor   = \FileAuthor,%
    pdftitle    = \FileTitle,%
    pdfsubject  = \FileSubject,%
    pdfkeywords = \FileKeyWords,%
    pdfcreator  = \LaTeX,%
    pdfproducer = \LaTeX}
\else
  \usepackage[dvips]{hyperref}
\fi

\definecolor{firstnamecolor}{RGB}{56,115,179}
\definecolor{familynamecolor}{RGB}{56,115,179}
\hypersetup{pdfborder=0 0 0}

\begin{document}
\sffamily   % for use with a résumé using sans serif fonts;
%\rmfamily  % for use with a résumé using serif fonts;
\hfill%
\begin{minipage}[t]{.6\textwidth}
	\raggedleft%
	{\bfseries {\color{firstnamecolor}\firstname}~{\color{familynamecolor}\familyname}}\\[.35ex]
	\small\itshape%
	Schwalbacher Straße 7\\
	12161 Berlin\\[.35ex]
	\Mobilefone~+49 162 32 39 557 \\
	\Letter~\href{mailto:pascal.bernhard@rppr.de}{pascal.bernhard@rppr.de}
\end{minipage}\\[0.5em]
%
{\color{firstnamecolor}\rule{\textwidth}{.25ex}}
%
\begin{minipage}[t]{.4\textwidth}
	\raggedright%
	% {\bfseries {\color{firstnamecolor}
	\vspace*{1em}
	\textbf{Miller \& Meier Consulting} \\
	Frau Constanze Miller \\[.35ex]
	% }}
	\small%
	Französische Straße 55\\
	10117 Berlin
\end{minipage}
%
\hfill
%
\begin{minipage}[t]{.4\textwidth}
	\raggedleft % US style
	\today
	%April 6, 2006 % US informal style
	%05/04/2006 % UK formal style
\end{minipage}\\[1em]


{\bfseries \color{familynamecolor}Associate Strategische Politikberatung}\\[0.75em]

Sehr geehrte Frau Miller,\\[0.5em]
%
was kann ich Ihnen als Associate in der strategischen Politikberatung anbieten? Als Politikwissenschaftler natürlich vielfältige Analyse-Skills, zugleich langjährige Auslandserfahrung in Frankreich und den USA, wo ich neue Kulturen, Menschen und Denkweisen kennengelernt habe.

Mein Politikstudium habe ich an der Freien Universität Berlin und Science Po Paris verfolgt, um die komplexen Zusammenhänge in unserer Welt verstehen zu können. Die Beziehungen zwischen Wirtschaft und Politik als Systeme und Prozesse mit eigner Logik zu analysieren, waren Kernstück des Studiums in Deutschland. In vier sehr praxis-orientierten Semestern an Science Po im MBA-Studiengang 'Finance et Strat\'{e}gie habe ich darüberhinaus unternehmerische Perspektiven und Lösungsansätze kennengelernt. Meinem Interesse an Politik bin ich an der Universität besonders im Bereich politischer Ökonomie nachgegangen. So habe ich in meiner Diplomarbeit über Energiepolitik im Baltikum das Zusammenspiel zwischen politischer Logik, wirtschaftlicher Rahmenbedingungen von Gasmärkten und Marktregulierung im Kontext internationaler Beziehungen untersucht.


Wieso ich meine Karriere in der strategischen Politikberatung weiterführen möchte? 


\subsection*{\textsf{Notizen zum Bewerbungsschreiben}}

\begin{itemize}

\item wirtschaftliche Interessen sollen in Politik (Sprache der Politik) übersetzt werden

\item Kunden sollen politische Logik verstehen und so die Rahmenbedingungen mitgestalten können

\item \textsl{'Wir alle möchten etwas bewegen und über den Tellerrand hinausschauen'}\\
\ding{225} \textbf{Aufenthalte im Ausland, als Entdeckung neuer Horizonte}

\item 

\item Ich arbeite mich gerne und schnell in neue Inhalte ein

\item Neue Herausforderungen jeden Tag, fachlich, methodisch und persönlich\\
\ding{225} \textbf{Verbindung zum Coaching herstellen}

\paragraph*{\textsf{Meine Stärken:}}

\begin{enumerate}

\item \textbf{hervorragender Universitätsabschluss}

\item \textbf{exzellente Sprachkenntnisse und umfassende Auslandserfahrung}

\item \textbf{analytische Fähigkeiten in Abschlussarbeit und Promotion unter Beweis gestellt}

\item \textbf{\textsl{Wie kann ich meine Verlässlichkeit und Leidenschaft unter Beweis stellen?}}

\end{enumerate}



\end{itemize}


\vspace{5cm}

Als Politikwissenschaftler habe ich ein ausgeprägtes Verständnis der komplexen Zusammenhängen zwischen politischen Entscheidungen und Märkten entwickelt. In meinen Tätigkeiten an der AmCham in Paris und der Unternehmensberatung SCI Verkehr in Berlin habe ich Entscheidungsprozesse in Wirtschaft und Politik analysiert sowie an quantitativen Marktstudien und Wettbewerbsanalysen im Verkehrs- und Infrastruktursektor mitgearbeitet. Dies beinhaltete die Ausarbeitung von Lösungen zur Ausrichtung der Unternehmensstrategie mit Berücksichtigung regulativer Risiken. An der Kommunikation dieser Vorschläge war ich ebenfalls aktiv beteiligt.

Bei SCI Verkehr konnte ich mir durch meine Mitarbeit an Studien zu Schienenverkehrsbetreibern und Begleitung von Ausschreibungsverfahren ein breites Wissen zum ÖPNV-Sektor in Deutschland aneignen. Als Assistant Policy Advisor an der AmCham in Paris habe ich dank sicherer Englisch- und Französischkenntnisse an Verhandlungen zu Unternehmenskooperationen teilgenommen.

Meinen eingeschlagenen Weg als analytisch denkender Problemlöser möchte ich nun, nach erfolgreichem Ende des Studiums, gerne zusammen mit Ihnen fortsetzen. Mich reizt es, meine Kompetenzen als Coach und Berater bei der gemeinsamen Entwicklung von Lösungen in Ihrer Unternehmensberatung einbringen und ausbauen zu können. Die im In- und Ausland erworbenen Erfahrungen bilden eine hervorragende Basis für eine Tätigkeit bei Ihnen.


Einem baldigen Einstieg bei Ihnen steht nichts entgegen.

Ich freue mich darauf, mehr über diese Stelle zu erfahren und Sie persönlich kennenzulernen.

  
%Yours sincerely,\\[2em] % if the opening is "Dear Mr(s) Doe,"
Mit freundlichen Grüßen,\\[3em] % if the opening is "Dear Sir or Madam,"
%
%\includegraphics[scale=0.75]{signature_blue}\\
{\bfseries \firstname~\familyname}\\
%
%\vfill%
%{\slshape \bfseries Bewerbungsunterlagen}\\
% {\slshape Lebenslauf\\
% Arbeitszeugnisse\\
% Diplomzeugnis{}}
\end{document}
