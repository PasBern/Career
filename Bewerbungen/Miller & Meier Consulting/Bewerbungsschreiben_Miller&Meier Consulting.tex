\documentclass[11pt,a4paper]{article}
\usepackage{ngerman}
\usepackage[ngerman]{babel}
\usepackage[utf8x]{inputenc}
\usepackage[T1]{fontenc}
\usepackage{lmodern}
\usepackage{marvosym}
\usepackage{amsfonts,amsmath,amssymb}
\usepackage{textcomp}
\usepackage{ifpdf}
\usepackage{pifont}
\usepackage[pdftex]{color}
\ifpdf
  \usepackage[pdftex]{graphicx}
\else
  \usepackage[dvips]{graphicx}\fi

\pagestyle{empty}

\usepackage[scale=0.775]{geometry}
\setlength{\parindent}{0pt}
\addtolength{\parskip}{6pt}

\def\firstname{Pascal}
\def\familyname{Bernhard}
\def\FileAuthor{\firstname~\familyname}
\def\FileTitle{\firstname~\familyname's Bewerbungsschreiben}
\def\FileSubject{Bewerbungsschreiben}
\def\FileKeyWords{\firstname~\familyname, Bewerbungsschreiben}

\renewcommand{\ttdefault}{pcr}
\hyphenation{ins-be-son-de-re}
\usepackage{url}
\urlstyle{tt}
\ifpdf
  \usepackage[pdftex,pdfborder=0,breaklinks,baseurl=http://,pdfpagemode=None,pdfstartview=XYZ,pdfstartpage=1]{hyperref}
  \hypersetup{
    pdfauthor   = \FileAuthor,%
    pdftitle    = \FileTitle,%
    pdfsubject  = \FileSubject,%
    pdfkeywords = \FileKeyWords,%
    pdfcreator  = \LaTeX,%
    pdfproducer = \LaTeX}
\else
  \usepackage[dvips]{hyperref}
\fi

\definecolor{firstnamecolor}{RGB}{56,115,179}
\definecolor{familynamecolor}{RGB}{56,115,179}
\hypersetup{pdfborder=0 0 0}

\begin{document}
\sffamily   % for use with a résumé using sans serif fonts;
%\rmfamily  % for use with a résumé using serif fonts;
\hfill%
\begin{minipage}[t]{.6\textwidth}
	\raggedleft%
	{\bfseries {\color{firstnamecolor}\firstname}~{\color{familynamecolor}\familyname}}\\[.35ex]
	\small\itshape%
	Schwalbacher Straße 7\\
	12161 Berlin\\[.35ex]
	\Mobilefone~+49 162 32 39 557 \\
	\Letter~\href{mailto:pascal.bernhard@rppr.de}{pascal.bernhard@rppr.de}
\end{minipage}\\[0.5em]
%
{\color{firstnamecolor}\rule{\textwidth}{.25ex}}
%
\begin{minipage}[t]{.4\textwidth}
	\raggedright%
	% {\bfseries {\color{firstnamecolor}
	\vspace*{1em}
	\textbf{Miller \& Meier Consulting} \\
	Frau Constanze Miller \\[.35ex]
	% }}
	\small%
	Französische Straße 55\\
	10117 Berlin
\end{minipage}
%
\hfill
%
\begin{minipage}[t]{.4\textwidth}
	\raggedleft % US style
	\today
	%April 6, 2006 % US informal style
	%05/04/2006 % UK formal style
\end{minipage}\\[1em]


{\bfseries \color{familynamecolor}Associate Strategische Politikberatung}\\[0.75em]

Sehr geehrte Frau Miller,\\[0.5em]
%

was ich kann Ihnen als Associate in der strategischen Politikberatung anbieten? Als Politikwissenschaftler natürlich vielfältige Analyse-Skills, zugleich langjährige Auslandserfahrung in Frankreich und den USA, wo ich neue Kulturen, Menschen und Denkweisen kennengelernt habe.

Mein Politikstudium habe ich an der Freien Universität Berlin und Science Po Paris verfolgt, um die komplexen Zusammenhänge in unserer Welt verstehen zu können. Die Beziehungen zwischen Wirtschaft und Politik als Systeme und Prozesse mit eigner Logik zu analysieren, waren Kernstück des Studiums in Deutschland. In vier sehr praxis-orientierten Semestern an Science Po im MBA-Studiengang 'Finance et Strat\'{e}gie habe ich darüberhinaus unternehmerische Perspektiven und Lösungsansätze kennengelernt. Meinem Interesse an Politik bin ich an der Universität besonders im Bereich politischer Ökonomie nachgegangen. So habe ich in meiner Diplomarbeit über Energiepolitik im Baltikum das Zusammenspiel zwischen politischer Logik, wirtschaftlicher Rahmenbedingungen von Gasmärkten und Marktregulierung im Kontext internationaler Beziehungen untersucht. \textsl{In meiner Promotion werde weiterverfolgen und geopgrafisch auf das Verhältnis zu Russland erweitern.}

Analytische Fähigkeiten im Consulting und ganz besonders im Bereich der Marktregulierung waren ich während meiner Arbeit bei der American Chamber auf Commerce in Paris und der Unternehmensberatung SCI Verkehr gefordert. Meine zentrale Aufgabe bestand darin, unseren Kunden ein Verständnis für die Logik von politischen Entscheidungsprozessen zu vermitteln und gemeinsam Lösungen für die Strategie des Unternehmens zu entwickeln. In meiner gegenwärtigen Tätigkeiten als Coach für Studenten kann ich die Fähigkeit mich schnell in neue Sachverhalte einzuarbeiten und Ideen weiterzugeben vereinen mit der Freude, Menschen zu helfen und sie zum Erfolg zu begleiten. Ausgehend von meinen eigenen Erfahrungen mit den Herausforderungen des selbstbestimmten Studiums unterstütze ich meine Klienten auf der einen Seite inhaltlich bei wissenschaftlichen Arbeiten und helfe zugleich mit Motivationsschwierigkeiten und Arbeitsorganisation.

Passe ich als Associate zu Ihnen in die strategischen Politikberatung? Ja, auf jeden Fall! Mit mir erhalten sie einen aufgeweckten Mitarbeiter, der nicht auf den Kopf gefallen ist -- komplexe Sachverhalte erfasse ich schnell und kann hieraus weitblickende Lösungen für das politische Umfeld erarbeiten und kommunizieren. Meine Karriere möchte ich bei Ihnen in der Politikberatung weiterführen, \textsl{um auch im direkten Kontakt mit Politik Energiethemen und Chancengleichheit in der Gesellschaft mitgestalten zu können.}

Einem baldigen Einstieg bei Ihnen steht nichts entgegen. Auf ein Kennenlernen freue ich mich, so kann ich Sie von meinen Qualifikation überzeugen und mehr über diese Stelle zu erfahren.


Mit freundlichen Grüßen,\\[2em] 
%
%\includegraphics[scale=0.75]{signature_blue}\\
{\bfseries \firstname~\familyname}\\
%
%\vfill%
%{\slshape \bfseries Bewerbungsunterlagen}\\
% {\slshape Lebenslauf\\
% Arbeitszeugnisse\\
% Diplomzeugnis{}}
\end{document}
