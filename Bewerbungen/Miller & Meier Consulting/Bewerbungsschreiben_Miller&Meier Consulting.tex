\documentclass[10pt,a4paper]{article}
\usepackage{ngerman}
\usepackage[ngerman]{babel}
\usepackage[utf8x]{inputenc}
\usepackage[T1]{fontenc}
\usepackage{lmodern}
\usepackage{marvosym}
\usepackage{amsfonts,amsmath,amssymb}
\usepackage{textcomp}
\usepackage{ifpdf}
\usepackage{pifont}
\usepackage[pdftex]{color}
\ifpdf
  \usepackage[pdftex]{graphicx}
\else
  \usepackage[dvips]{graphicx}\fi

\pagestyle{empty}

\usepackage[top=1.2cm,bottom=1.2cm,scale=0.775]{geometry}
\setlength{\parindent}{0pt}
\addtolength{\parskip}{6pt}

\def\firstname{Pascal}
\def\familyname{Bernhard}
\def\FileAuthor{\firstname~\familyname}
\def\FileTitle{\firstname~\familyname's Bewerbungsschreiben}
\def\FileSubject{Bewerbungsschreiben}
\def\FileKeyWords{\firstname~\familyname, Bewerbungsschreiben}

\renewcommand{\ttdefault}{pcr}
\hyphenation{ins-be-son-de-re}
\usepackage{url}
\urlstyle{tt}
\ifpdf
  \usepackage[pdftex,pdfborder=0,breaklinks,baseurl=http://,pdfpagemode=None,pdfstartview=XYZ,pdfstartpage=1]{hyperref}
  \hypersetup{
    pdfauthor   = \FileAuthor,%
    pdftitle    = \FileTitle,%
    pdfsubject  = \FileSubject,%
    pdfkeywords = \FileKeyWords,%
    pdfcreator  = \LaTeX,%
    pdfproducer = \LaTeX}
\else
  \usepackage[dvips]{hyperref}
\fi

\definecolor{firstnamecolor}{RGB}{56,115,179}
\definecolor{familynamecolor}{RGB}{56,115,179}
\hypersetup{pdfborder=0 0 0}

\begin{document}
\sffamily   % for use with a résumé using sans serif fonts;
%\rmfamily  % for use with a résumé using serif fonts;
\hfill%
\begin{minipage}[t]{.6\textwidth}
	\raggedleft%
	{\bfseries {\color{firstnamecolor}\firstname}~{\color{familynamecolor}\familyname}}\\[.35ex]
	\small\itshape%
	Schwalbacher Straße 7\\
	12161 Berlin\\[.35ex]
	\Mobilefone~+49 162 32 39 557 \\
	\Letter~\href{mailto:pascal.bernhard@rppr.de}{pascal.bernhard@rppr.de}
\end{minipage}\\[0.5em]
%
{\color{firstnamecolor}\rule{\textwidth}{.25ex}}
%
\begin{minipage}[t]{.4\textwidth}
	\raggedright%
	% {\bfseries {\color{firstnamecolor}
	\vspace*{1em}
	\textbf{Verein Deutscher Zementwerke} \\
	 Herr Dr. Martin Schneider\\[.35ex]
	% }}
	\small%
	Postfach 30 10 63\\
	40410 Düsseldorf
\end{minipage}
%
\hfill
%
\begin{minipage}[t]{.4\textwidth}
	\raggedleft % US style
%	\today
	%April 6, 2006 % US informal style
	%05/04/2006 % UK formal style
\end{minipage}\\[1em]
{\bfseries \color{familynamecolor}Referent für Energie- und Klimapolitik}\\[0.75em]

Sehr geehrte Herr Dr. Schneider,\\[0.5em]
%
bei meiner Suche nach einer neuen herausfordernden Tätigkeit bin ich bei Stepstone auf Ihr Stellenangebot eines Referenten getroffen. Als Politikwissenschaftler mit vielfältigen Analyse-Skills sehe ich in meinem Spezialisierungsbereich Energie- und Klimapolitik eine hervorragende Chance, meine Fähigkeiten einzubringen und mich persönlich weiterzuentwickeln.


Die Grundlagen meiner analytischen Fähigkeiten habe ich in meinem Studium in Deutschland und Frankreich gelernt. Politik in Berlin studiert, um komplexe Zusammenhänge in unserer Welt besser verstehen zu können. 


Mein Politikstudium habe ich an der Freien Universität Berlin und Science Po Paris verfolgt, um die komplexen Zusammenhänge in unserer Welt verstehen zu können. In vier sehr praxisnahen Semester an Science Po Paris im MBA-Studiengang 'Finance et Strat\'{e}gie konnte ich sehr weit über den deutschen politischen Tellerrand hinausschauen. Einblicke in betriebswirtschaftliche Abläufe gaben mir Module zu Marktstrategien, Rechnungswesen und internationalem Handelsrecht und mir ein bessere Verständnis für die spezifischen Anliegen der Unternehmen vermittelt. Neugierde geweckt an der Vermittlung wirtschaftlicher Interessen an die Politik habe ich mich im Hauptstudium an der FU Berlin auf politische Ökonomie und Marktregulierung konzentriert.


Meine ausgeprägten analytischen Fähigkeiten konnte ich in meiner Abschlussarbeit und momentan in der Promotion unter Beweis stellen. Entwickeln konnte ich diese Kompetenzen in Praktika an der American Chamber of Commerce und SCI Verkehr wie auch meiner Berufspraxis als Coach. Bei meinen Analyse-Arbeiten zu politischen Rahmenbedingungen für Investitionsentscheidungen habe ich schätzen gelernt, wie fruchtbar der Austausch mit Kollegen unterschiedlichen fachlichen Hintergrundes ist. Dank sicherere Englisch- und Französischkenntnisse habe ich zudem aktiv an der Vermittlung von internationalen Unternehmenskooperationen für Joint Ventures und Syndizierung von Venture Capital-Projekten teilgenommen.

Meine jetzige Tätigkeit als Coach für Studenten stellt mich täglich vor die spannende Herausforderung mich in neue Sachverhalte einzuarbeiten. Studenten helfe ich dabei Motivationsprobleme zu überwinden und strukturiert zu arbeiten, so dass sie trotz häufig schwieriger sozialer Umstände ihr Studium erfolgreich beenden können. Dabei kommt mir sowohl mein Fachwissen als auch meine natürliche Neugierde zu Gute, die mich schon immer bewogen haben, an Projekten Interesse zu zeigen, die etwas mehr an Gerechtigkeit in diese Welt bringen wollen. Den Wunsch soziale und politische Gegebenheit gerechter zu gestalten treibt mich auch an, im Ruderverein Wiking ein Projekt zur Integration benachteiligter Jugendlicher mitaufzubauen. Hier sehe ich meine Aufgabe darin, dank meiner Kenntnisse zu Entscheidungsprozessen im Berliner Senat, Möglichkeiten politischer Unterstützung für unsere Maßnahmen zu identifizieren und einzuwerben.

Passe ich als Associate zu Ihnen in die strategischen Politikberatung? Ja, auf jeden Fall! Mit mir erhalten sie einen aufgeweckten Mitarbeiter, der nicht auf den Kopf gefallen ist -- komplexe Sachverhalte erfasse ich schnell und kann weitblickende Lösungen für das politische Umfeld erarbeiten und kommunizieren. Es reizt mich, meine Kompetenzen als Berater und Coach bei der gemeinsamen Entwicklung von Lösungen in Ihrer Unternehmensberatung einbringen und ausbauen zu können.


Einem baldigen Einstieg bei Ihnen steht nichts entgegen. Ich freue mich darauf, mehr über diese Stelle zu erfahren und Sie persönlich kennenzulernen.


Mit freundlichen Grüßen,\\[2em] 
%
\includegraphics[scale=0.15]{Unterschrift.png}\\
{\bfseries {\footnotesize{\firstname~\familyname}}\\
%
%\vfill%
%{\slshape \bfseries Bewerbungsunterlagen}\\
% {\slshape Lebenslauf\\
% Arbeitszeugnisse\\
% Diplomzeugnis{}}
\end{document}


Meinem Interesse an Politik bin ich an der Universität besonders im Bereich politischer Ökonomie nachgegangen. 

Stets lag die zentrale Aufgabe darin, unseren Kunden ein Verständnis für die Logik dieser Entscheidungsprozessen zu vermitteln und gemeinsam Lösungen zu entwickeln, wie Unternehmensinteressen hierbei vertreten werden können.