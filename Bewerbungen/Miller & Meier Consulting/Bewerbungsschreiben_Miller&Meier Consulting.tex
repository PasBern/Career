\documentclass[11pt,a4paper]{article}
\usepackage{ngerman}
\usepackage[ngerman]{babel}
\usepackage[utf8x]{inputenc}
\usepackage[T1]{fontenc}
\usepackage{lmodern}
\usepackage{marvosym}
\usepackage{amsfonts,amsmath,amssymb}
\usepackage{textcomp}
\usepackage{ifpdf}
\usepackage{pifont}
\usepackage[pdftex]{color}
\ifpdf
  \usepackage[pdftex]{graphicx}
\else
  \usepackage[dvips]{graphicx}\fi

\pagestyle{empty}

\usepackage[scale=0.775]{geometry}
\setlength{\parindent}{0pt}
\addtolength{\parskip}{6pt}

\def\firstname{Pascal}
\def\familyname{Bernhard}
\def\FileAuthor{\firstname~\familyname}
\def\FileTitle{\firstname~\familyname's Bewerbungsschreiben}
\def\FileSubject{Bewerbungsschreiben}
\def\FileKeyWords{\firstname~\familyname, Bewerbungsschreiben}

\renewcommand{\ttdefault}{pcr}
\hyphenation{ins-be-son-de-re}
\usepackage{url}
\urlstyle{tt}
\ifpdf
  \usepackage[pdftex,pdfborder=0,breaklinks,baseurl=http://,pdfpagemode=None,pdfstartview=XYZ,pdfstartpage=1]{hyperref}
  \hypersetup{
    pdfauthor   = \FileAuthor,%
    pdftitle    = \FileTitle,%
    pdfsubject  = \FileSubject,%
    pdfkeywords = \FileKeyWords,%
    pdfcreator  = \LaTeX,%
    pdfproducer = \LaTeX}
\else
  \usepackage[dvips]{hyperref}
\fi

\definecolor{firstnamecolor}{RGB}{56,115,179}
\definecolor{familynamecolor}{RGB}{56,115,179}
\hypersetup{pdfborder=0 0 0}

\begin{document}
\sffamily   % for use with a résumé using sans serif fonts;
%\rmfamily  % for use with a résumé using serif fonts;
\hfill%
\begin{minipage}[t]{.6\textwidth}
	\raggedleft%
	{\bfseries {\color{firstnamecolor}\firstname}~{\color{familynamecolor}\familyname}}\\[.35ex]
	\small\itshape%
	Schwalbacher Straße 7\\
	12161 Berlin\\[.35ex]
	\Mobilefone~+49 162 32 39 557 \\
	\Letter~\href{mailto:pascal.bernhard@rppr.de}{pascal.bernhard@rppr.de}
\end{minipage}\\[0.5em]
%
{\color{firstnamecolor}\rule{\textwidth}{.25ex}}
%
\begin{minipage}[t]{.4\textwidth}
	\raggedright%
	% {\bfseries {\color{firstnamecolor}
	\vspace*{1em}
	\textbf{Miller \& Meier Consulting} \\
	Frau Constanze Miller \\[.35ex]
	% }}
	\small%
	Französische Straße 55\\
	10117 Berlin
\end{minipage}
%
\hfill
%
\begin{minipage}[t]{.4\textwidth}
	\raggedleft % US style
	\today
	%April 6, 2006 % US informal style
	%05/04/2006 % UK formal style
\end{minipage}\\[1em]


{\bfseries \color{familynamecolor}Associate Strategische Politikberatung}\\[0.75em]

Sehr geehrte Frau Miller,\\[0.5em]
%
Ihre Stellenanzeige für einen Associate, der sie unterstützt, Lösungsansätze für komplexe Fragestellung zu erarbeiten, hat umgehend meine Interesse geweckt. Als Politikwissenschaftler mit vielfältigen Analyse-Skills sehe ich in der strategischen Politikberatung eine hervorragende Chance, meine Fähigkeiten einzubringen und mich persönlich weiterzuentwickeln.

Mein Politikstudium habe ich an der Freien Universität Berlin und Science Po Paris verfolgt, um die komplexen Zusammenhänge in unserer Welt verstehen zu können. Die vier sehr praxis-nahen Semester an Science Po im MBA-Studiengang 'Finance et Strat\'{e}gie konnte ich sehr weit über den deutschen politischen Tellerrand hinausschauen. In den Modulen zu Marktstrategien, Rechnungswesen und internationalem Vertragsrecht habe Einblicke in betriebswirtschaftliche Abläufe gewonnen und verstehen gelernt, welche Interessen Unternehmen gegenüber Politik haben. Meine Neugierde geweckt am Zusammenspiel zwischen politischer Logik und Marktgeschehen habe ich mich im Hauptstudium an der FU Berlin auf politische Ökonomie und Marktregulierung konzentriert.


Meine ausgeprägten analytischen Fähigkeiten konnte ich in meiner Abschlussarbeit zur Energiemarktliberalisierung im Baltikum und momentan in der Promotion zu Energiepolitik in Osteuropa unter Beweis stellen. Aber auch schon in meiner Praktika an der American Chamber of Commerce und SCI Verkehr wie auch meiner Berufspraxis als Coach habe ich diese Kompetenzen demonstriert. An der amerikanischen Handelskammer in Paris war ich gefordert erst unterstützend dann als Assistant Policy Advisor eigenständig Analysen der politischen Rahmenbedingungen für Investitionsentscheidungen im Infrastruktur- und Utilitiessektor zu erstellen. Bei der Unternehmensberatung SCI Verkehr waren die Konsequenzen politischer Beschlüsse und Marktregulierung für Unternehmen im Schienenverkehrsbereich im Fokus meiner Arbeit. Dank sicherere Englisch- und Französischkenntnisse habe ich zudem aktiv an Verhandlungen zu internationalen Unternehmenskooperationen für Joint Ventures und Syndizierung von Venture Capital-Projekten teilgenommen.

Meine jetzige Tätigkeit als Coach für Studenten stellt mich täglich vor die erfrischende Herausforderung mich in neue Sachverhalte einzuarbeiten. Den Studenten helfe ich dabei Motivationsprobleme zu überwinden und strukturiert zu arbeiten, so dass sie trotz häufig schwieriger sozialer Umstände ihr Studium erfolgreich beenden können. Dabei kommt mir sowohl mein Fachwissen als auch meine natürliche Neugierde zu Gute, die mich schon immer bewogen haben, an Projekten Interesse zu zeigen, die etwas mehr an Gerechtigkeit in diese Welt bringen wollen. Den Wunsch soziale und politische Gegebenheit gerechter zu gestalten treibt mich auch an im Ruderverein Wiking ein Projekt zur Integration benachteiligter Jugendlicher mitaufzubauen.

Passe ich als Associate zu Ihnen in die strategischen Politikberatung? Ja, auf jeden Fall! Mit mir erhalten sie einen aufgeweckten Mitarbeiter, der nicht auf den Kopf gefallen ist -- komplexe Sachverhalte erfasse ich schnell und kann hieraus weitblickende Lösungen für das politische Umfeld erarbeiten und kommunizieren. Es reizt mich, meine Kompetenzen als Coach und Berater bei der gemeinsamen Entwicklung von Lösungen in Ihrer Unternehmensberatung einbringen und ausbauen zu können.


Einem baldigen Einstieg bei Ihnen steht nichts entgegen. Auf ein Kennenlernen freue ich mich, so kann ich Sie von meinen Qualifikation überzeugen und mehr über diese Stelle zu erfahren.


Mit freundlichen Grüßen,\\[2em] 
%
%\includegraphics[scale=0.75]{signature_blue}\\
{\bfseries \firstname~\familyname}\\
%
%\vfill%
%{\slshape \bfseries Bewerbungsunterlagen}\\
% {\slshape Lebenslauf\\
% Arbeitszeugnisse\\
% Diplomzeugnis{}}
\end{document}


Meinem Interesse an Politik bin ich an der Universität besonders im Bereich politischer Ökonomie nachgegangen. 

Stets lag die zentrale Aufgabe darin, unseren Kunden ein Verständnis für die Logik dieser Entscheidungsprozessen zu vermitteln und gemeinsam Lösungen zu entwickeln, wie Unternehmensinteressen hierbei vertreten werden können.