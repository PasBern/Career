\documentclass[10pt,a4paper]{article}
\usepackage{ngerman}
\usepackage[ngerman]{babel}
\usepackage[utf8x]{inputenc}
\usepackage[T1]{fontenc}
\usepackage{lmodern}
\usepackage{marvosym}
\usepackage{amsfonts,amsmath,amssymb}
\usepackage{textcomp}
\usepackage{ifpdf}
\usepackage{pifont}
\usepackage[pdftex]{color}
\ifpdf
  \usepackage[pdftex]{graphicx}
\else
  \usepackage[dvips]{graphicx}\fi

\pagestyle{empty}

\usepackage[top=1.2cm,bottom=1.2cm,scale=0.775]{geometry}
\setlength{\parindent}{0pt}
\addtolength{\parskip}{6pt}

\def\firstname{Pascal}
\def\familyname{Bernhard}
\def\FileAuthor{\firstname~\familyname}
\def\FileTitle{\firstname~\familyname's Bewerbungsschreiben}
\def\FileSubject{Bewerbungsschreiben}
\def\FileKeyWords{\firstname~\familyname, Bewerbungsschreiben}

\renewcommand{\ttdefault}{pcr}
\hyphenation{ins-be-son-de-re}
\usepackage{url}
\urlstyle{tt}
\ifpdf
  \usepackage[pdftex,pdfborder=0,breaklinks,baseurl=http://,pdfpagemode=None,pdfstartview=XYZ,pdfstartpage=1]{hyperref}
  \hypersetup{
    pdfauthor   = \FileAuthor,%
    pdftitle    = \FileTitle,%
    pdfsubject  = \FileSubject,%
    pdfkeywords = \FileKeyWords,%
    pdfcreator  = \LaTeX,%
    pdfproducer = \LaTeX}
\else
  \usepackage[dvips]{hyperref}
\fi

\definecolor{firstnamecolor}{RGB}{56,115,179}
\definecolor{familynamecolor}{RGB}{56,115,179}
\hypersetup{pdfborder=0 0 0}

\begin{document}
\sffamily   % for use with a résumé using sans serif fonts;
%\rmfamily  % for use with a résumé using serif fonts;
\hfill%
\begin{minipage}[t]{.6\textwidth}
	\raggedleft%
	{\bfseries {\color{firstnamecolor}\firstname}~{\color{familynamecolor}\familyname}}\\[.35ex]
	\small\itshape%
	Schwalbacher Straße 7\\
	12161 Berlin\\[.35ex]
	\Mobilefone~+49 162 32 39 557 \\
	\Letter~\href{mailto:pascal.bernhard@rppr.de}{pascal.bernhard@rppr.de}
\end{minipage}\\[0.5em]
%
{\color{firstnamecolor}\rule{\textwidth}{.25ex}}

\hfill

\begin{flushleft}
{\bfseries Motivationsschreiben Strategy School -- Workshop Car Sharing}\\[0.75em]
\end{flushleft}

Sehr geehrte Damen und Herren,\\[0.5em]
%

Über die digitale Zukunft unserer Wirtschaft zu lesen und zu diskutieren finde ich bereits interessant, aber die Herausforderung, selbst Strategien hierfür zu entwickeln, fasziniert mich noch mehr. Ich will mehr als nur passiv Veränderungen in unserer Gesellschaft beobachten, ich will aktiv mitgestalten können, denn ich sehe mich als jemand der anpackt, nicht zuschaut. Daher hat mich Ihr Angebot einer Strategy School zur Digitalisierung auf Anhieb aufmerksam gemacht. Hier in den Workshops will ich meine Ideen einbringen, wie Unternehmen mit der zukünftigen Vielfalt an Informationen ihr Produktportfolio genauer auf einzelne Kundengruppen ausrichten können.

Was bringe ich als Politikwissenschaftler in die Workshops ein? Die Zusammenhänge zwischen Politik und Wirtschaft zu analysieren zählt zu meinen größten Stärken. Den Blick für die politischen Spielregeln der Wirtschaft habe ich in meinem Politikstudium an der FU Berlin schärfen können. Vier Semester an der Fondation Nationale des Sciences Politiques im Master 'Finance et Stratégie', das war die Einführung in Unternehmens- und Produktstrategien. An den dortigen Fallstudien reizte mich besonders, interdisziplinär mit anderen Studenten Ideen und Pläne für die Positionierung unter ständig veränderten Marktbedingungen zu entwerfen. Hierbei konnten wir viel von den Erfahrungen der Dozenten aus ihrer Berufspraxis lernen.

Da mir diese Arbeit sehr viel Freude bereitet, habe ich parallel zum Studium als Praktikant an der Amerikanischen Handelskammer in Paris die Möglichkeiten von Private Public Partnerships auf dem französischen Infrastrukturmarkt analysiert. Dank meiner selbständigen Arbeitsweise und dem souveränem Umgang mit Fremdsprachen und anderen Kulturen konnte ich im folgenden Jahr die Handelskammer als Assistant Policy Advisor Europe bei Verhandlungen zu internationalen Joint Ventures unterstützen. Hier habe ich unterschiedliche Unternehmenskulturen kennengelernt und eine Vielzahl an Kontakten geknüpft.

So habe ich nicht nur Bücher gelesen und Vorlesungen besucht, ich habe mich auch mit Menschen auseinandergesetzt. Die hieraus entwickelte Fähigkeit, mich gut auf andere Personen einstellen zu können, hilft mir dabei, produktiv in der Teamarbeit mitzuwirken. Meine jetzige Tätigkeit als Coach für Studenten reizt mich besonders, weil ich abwechslungsreiche Aufgaben mag. Ich unterstütze mit Einfühlungsvermögen Studenten dabei, strukturiert zu arbeiten und auch angesichts manchmal schwieriger sozialer Umstände und Motivationsschwierigkeiten ihr Studium zu einem gelungenen Abschluss zu führen. So haben mit meiner Hilfe alle meine Klienten erfolgreiche Abschlussarbeiten geschrieben. Ich habe Freude an meiner Arbeit, anderen Menschen zu helfen, und möchte zugleich auch meinen Wunsch, beruflich eigene Ideen zu verwirklichen, weiterverfolgen.

Ich strebe für meine weitere Laufbahn mittelfristig eine unternehmerische Tätigkeit an. In meiner Karrieregestaltung bin ich flexibel, um angesichts unserer sich stets weiterentwickelnden Wirtschaft passend aufgestellt zu sein. Entweder möchte ich ein Start-Up beim Unternehmensaufbau mit meinem Wissen zu Marktregulierung begleiten oder selbst ein Unternehmen im Bereich Beratung gründen. 

Als Berliner erlebe ich jeden Tag, vor welchen Herausforderungen der Großstadtverkehr steht, und mache mir Gedanken über bessere Lösungen für die Zukunft. Hier sehe ich großes Potential in einem Car Sharing, welches dank dem Internet-of-Things und moderner Antriebstechnologien Flexibilität und Umweltbewusstsein für die Kunden vereinen kann. Im Workshop will ich erfahren, wie ich im Team für diese Idee ein tragfähiges Start-Up -- Konzept entwickeln kann. 

\vspace{0.5cm}

Mit freundlichen Grüßen,\\[1em] 
%
\includegraphics[scale=0.15]{Unterschrift.png}\\
{\bfseries {\footnotesize{\firstname~\familyname}}\\

\end{document}

% Zusätzliche Notizen

Neue Wege bin ich auch nach dem Abitur gegangen, als ich als Kind aus der Stadt zuerst ein Jahr lang in Frankreich auf dem Bauernhof gearbeit habe, dann ein weiteres Jahr wieder in den USA im Bauunternehmen von Freunden als Dispatcher Genehmigungsverfahren begleitet und bei der Büroorganisation erste Berufserfahrungen gemacht habe.
