\documentclass[10pt,a4paper]{article}
\usepackage{ngerman}
\usepackage[ngerman]{babel}
\usepackage[utf8x]{inputenc}
\usepackage[T1]{fontenc}
\usepackage{lmodern}
\usepackage{marvosym}
\usepackage{amsfonts,amsmath,amssymb}
\usepackage{textcomp}
\usepackage{ifpdf}
\usepackage{pifont}
\usepackage[pdftex]{color}
\ifpdf
  \usepackage[pdftex]{graphicx}
\else
  \usepackage[dvips]{graphicx}\fi

\pagestyle{empty}

\usepackage[top=1.2cm,bottom=1.2cm,scale=0.775]{geometry}
\setlength{\parindent}{0pt}
\addtolength{\parskip}{6pt}

\def\firstname{Pascal}
\def\familyname{Bernhard}
\def\FileAuthor{\firstname~\familyname}
\def\FileTitle{\firstname~\familyname's Bewerbungsschreiben}
\def\FileSubject{Bewerbungsschreiben}
\def\FileKeyWords{\firstname~\familyname, Bewerbungsschreiben}

\renewcommand{\ttdefault}{pcr}
\hyphenation{ins-be-son-de-re}
\usepackage{url}
\urlstyle{tt}
\ifpdf
  \usepackage[pdftex,pdfborder=0,breaklinks,baseurl=http://,pdfpagemode=None,pdfstartview=XYZ,pdfstartpage=1]{hyperref}
  \hypersetup{
    pdfauthor   = \FileAuthor,%
    pdftitle    = \FileTitle,%
    pdfsubject  = \FileSubject,%
    pdfkeywords = \FileKeyWords,%
    pdfcreator  = \LaTeX,%
    pdfproducer = \LaTeX}
\else
  \usepackage[dvips]{hyperref}
\fi

\definecolor{firstnamecolor}{RGB}{56,115,179}
\definecolor{familynamecolor}{RGB}{56,115,179}
\hypersetup{pdfborder=0 0 0}

\begin{document}
\sffamily   % for use with a résumé using sans serif fonts;
%\rmfamily  % for use with a résumé using serif fonts;
\hfill%
\begin{minipage}[t]{.6\textwidth}
	\raggedleft%
	{\bfseries {\color{firstnamecolor}\firstname}~{\color{familynamecolor}\familyname}}\\[.35ex]
	\small\itshape%
	Schwalbacher Straße 7\\
	12161 Berlin\\[.35ex]
	\Mobilefone~+49 162 32 39 557 \\
	\Letter~\href{mailto:pascal.bernhard@rppr.de}{pascal.bernhard@rppr.de}
\end{minipage}\\[0.5em]
%
{\color{firstnamecolor}\rule{\textwidth}{.25ex}}
%
\begin{minipage}[t]{.4\textwidth}
	\raggedright%
	% {\bfseries {\color{firstnamecolor}
	\vspace*{1em}
	\textbf{Verein Deutscher Zementwerke} \\
	 Herr Dr. Martin Schneider\\[.35ex]
	% }}
	\small%
	Postfach 30 10 63\\
	40410 Düsseldorf
\end{minipage}
%
\hfill
%
\begin{minipage}[t]{.4\textwidth}
	\raggedleft % US style
	\today
	%April 6, 2006 % US informal style
	%05/04/2006 % UK formal style
\end{minipage}\\[1em]
{\bfseries \color{familynamecolor}Referent für Energie- und Klimapolitik}\\[0.75em]

Sehr geehrte Herr Dr. Schneider,\\[0.5em]
%
bei meiner Suche nach einer neuen herausfordernden Tätigkeit bin ich bei Stepstone auf Ihr Stellenangebot für einen Referenten getroffen. Als Politikwissenschaftler mit vielfältigen Analyse-Skills sehe ich in meinem Spezialgebiet der Energie- und Klimafragen eine hervorragende Chance, meine Fähigkeiten einzubringen und mich persönlich weiterzuentwickeln.


Während meines Studiums der Politikwissenschaft habe ich mich auf Energie - und Klimapolitik  spezialisiert. Im Rahmen meiner Diplomarbeit habe ich mit den politischen Kontext der Gasmarktliberalisierung in Europa beschäftigt und die Entwicklung der EU-Energiepolitik genau verfolgt. Zusätzlich zum Politikstudium wollte ich im MBA-Studiengang 'Finance et Strat\'{e}gie an der Pariser Fondation Nationale des Sciences Politiques mein Verständnis wirtschaftlicher Zusammenhänge erweitern. Die sehr praxisnahen Module zu Marktstrategien, Rechnungswesen und Unternehmensfinanzierung haben mich mit den Herausforderungen, denen sich Unternehmen am Markt gegenübersehen, vertraut gemacht.


Meine analytischen Kompetenzen konnte ich in Tätigkeiten an der American Chamber of Commerce und SCI Verkehr wie auch meiner Berufspraxis als Coach entwickeln. Bei meinen Arbeiten zu politischen Rahmenbedingungen für Investitionsentscheidungen habe ich schätzen gelernt, wie fruchtbar der Austausch mit Kollegen unterschiedlichen fachlichen Hintergrundes ist. Dank sicherer Englisch- und Französischkenntnisse habe ich zudem aktiv am Aufbau von internationalen Joint Ventures im Versorgungssektor und der Syndizierung von Venture Capital-Projekten teilgenommen.

Meine jetzige Tätigkeit als Coach für Studenten stellt mich vor die spannende Herausforderung, mich immer wieder in neue Sachverhalte einzuarbeiten. Studenten helfe ich dabei, Motivationsprobleme zu überwinden sowie strukturiert und effizient zu arbeiten, um auch angesichts teils schwierigen sozialen Umständen das Studium erfolgreich beenden zu können. Dabei kommt mir sowohl mein Fachwissen als auch meine natürliche Neugierde zu Gute, die mich schon immer bewogen haben, an Projekten Interesse zu zeigen, die etwas in dieser Welt zum Positiven hin verändern wollen. Dieses Anliegen treibt mich auch an, im Ruderverein Wiking ein Projekt zur Integration benachteiligter Jugendlicher mit aufzubauen. Hier bestehen meine Aufgaben darin, dank meiner Kenntnisse zu Entscheidungsprozessen in der Politik, Möglichkeiten politischer Unterstützung für unsere Maßnahmen zu identifizieren und einzuwerben.

Mit mir als Referenten erhalten sie einen aufgeweckten Mitarbeiter, der nicht auf den Kopf gefallen ist -- komplexe Sachverhalte erfasse ich schnell und kann weitblickende Lösungen für das politische Umfeld erarbeiten und kommunizieren. Es reizt mich, meine Kompetenzen als Berater und Coach bei der gemeinsamen Entwicklung von Lösungen in Ihrer Organisation einbringen und ausbauen zu können.


Einem baldigen Einstieg bei Ihnen steht nichts entgegen. Ich freue mich darauf, mehr über diese Stelle zu erfahren und Sie persönlich kennenzulernen.


Mit freundlichen Grüßen,\\[2em] 
%
\includegraphics[scale=0.15]{Unterschrift.png}\\
{\bfseries {\footnotesize{\firstname~\familyname}}\\
%
%\vfill%
%{\slshape \bfseries Bewerbungsunterlagen}\\
% {\slshape Lebenslauf\\
% Arbeitszeugnisse\\
% Diplomzeugnis{}}
\end{document}


Meinem Interesse an Politik bin ich an der Universität besonders im Bereich politischer Ökonomie nachgegangen. 

Stets lag die zentrale Aufgabe darin, unseren Kunden ein Verständnis für die Logik dieser Entscheidungsprozessen zu vermitteln und gemeinsam Lösungen zu entwickeln, wie Unternehmensinteressen hierbei vertreten werden können.