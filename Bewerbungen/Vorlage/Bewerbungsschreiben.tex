\documentclass[11pt,a4paper]{article}
\usepackage{ngerman}
\usepackage[ngerman]{babel}
\usepackage[utf8x]{inputenc}
\usepackage[T1]{fontenc}
\usepackage{lmodern}
\usepackage{marvosym}
\usepackage{ifpdf}
\usepackage[pdftex]{color}
\ifpdf
  \usepackage[pdftex]{graphicx}
\else
  \usepackage[dvips]{graphicx}\fi

\pagestyle{empty}

\usepackage[scale=0.775]{geometry}
\setlength{\parindent}{0pt}
\addtolength{\parskip}{6pt}

\def\firstname{Pascal}
\def\familyname{Bernhard}
\def\FileAuthor{\firstname~\familyname}
\def\FileTitle{\firstname~\familyname's Bewerbungsschreiben}
\def\FileSubject{Bewerbungsschreiben}
\def\FileKeyWords{\firstname~\familyname, Bewerbungsschreiben}

\renewcommand{\ttdefault}{pcr}
\hyphenation{ins-be-son-de-re}
\usepackage{url}
\urlstyle{tt}
\ifpdf
  \usepackage[pdftex,pdfborder=0,breaklinks,baseurl=http://,pdfpagemode=None,pdfstartview=XYZ,pdfstartpage=1]{hyperref}
  \hypersetup{
    pdfauthor   = \FileAuthor,%
    pdftitle    = \FileTitle,%
    pdfsubject  = \FileSubject,%
    pdfkeywords = \FileKeyWords,%
    pdfcreator  = \LaTeX,%
    pdfproducer = \LaTeX}
\else
  \usepackage[dvips]{hyperref}
\fi

\definecolor{firstnamecolor}{RGB}{56,115,179}
\definecolor{familynamecolor}{RGB}{56,115,179}
\hypersetup{pdfborder=0 0 0}

\begin{document}
\sffamily   % for use with a résumé using sans serif fonts;
%\rmfamily  % for use with a résumé using serif fonts;
\hfill%
\begin{minipage}[t]{.6\textwidth}
	\raggedleft%
	{\bfseries {\color{firstnamecolor}\firstname}~{\color{familynamecolor}\familyname}}\\[.35ex]
	\small\itshape%
	Schwalbacher Straße 7\\
	12161 Berlin\\[.35ex]
	\Mobilefone~+49 152 38 50 23 63 \\
	\Letter~\href{mailto:pbernhard@seitenfreunde.de}{pbernhard@seitenfreunde.de}
\end{minipage}\\[0.5em]
%
{\color{firstnamecolor}\rule{\textwidth}{.25ex}}
%
\begin{minipage}[t]{.4\textwidth}
	\raggedright%
	% {\bfseries {\color{firstnamecolor}
	\vspace*{1em}
	UNTERNEHMEN \\
	ANSPRECHPARTNERIN\\[.35ex]
	% }}
	\small%
	STRASSE\\
	PLZ STADT
\end{minipage}
%
\hfill
%
\begin{minipage}[t]{.4\textwidth}
	\raggedleft % US style
	\today
	%April 6, 2006 % US informal style
	%05/04/2006 % UK formal style
\end{minipage}\\[1em]
\raggedright

{\bfseries \color{familynamecolor}Bewerbung als Berater}\\[1.5em]

Sehr geehrte Frau NAME,\\[1em]
%
weshalb bin ich als Politikwissenschaftler die ideale Verstärkung für Ihr Team 'Wirtschaft, Energie, Infrastruktur'? Ihrer Unternehmensberatung biete ich vielfältige Analyse-Skills kombiniert mit langjähriger Auslandserfahrung sowie verhandlungssicherem Englisch und Französisch an. Dass Sie in Ihrem Stellenangebot, nach fundierter wissenschaftlicher Arbeitsweise und Erfahrungen in den Bereichen Energie und Infrastruktur suche, hat mich besonders angesprochen.


Mein Studium habe ich in Berlin und Paris absolviert. Systeme und Prozesse in Wirtschaft und Politik zu analysieren, waren Kernstück des Studiums an der Freien Universität. In vier sehr praxis-orientierten Semestern an Sciences Po Paris im MBA-Studiengang "Finance et Stratégie" habe ich zusätzlich zur akademischen Perspektive unternehmerische Denkweisen und Lösungsansätze kennengelernt. Verhandlungssichere Französisch- und Englischkenntnisse kann ich in Ihrer Unternehmensberatung dank mehrjähriger Auslandsaufenthalte in Frankreich und den USA einbringen.


Analytische Fähigkeiten in den Bereichen Consulting und Marktrecherchen waren während meiner Arbeit bei der American Chamber of Commerce in Paris und der Unternehmensberatung SCI Verkehr gefordert. Aus diesen Informationen Lösungen für unsere Kunden zu entwickeln, zählte dort zu unseren vorrangigen Aufgaben. Die Vorbereitung von Meetings und interne Arbeitsplanung gehörte ebenso zu meinem Aufgabengebiet. Auch im Politik-Studium war die Fähigkeit gefragt, Ergebnisse präzise zu dokumentieren und sie schriftlich wie auch in Präsentationen zu vermitteln. 

Meine gegenwärtige Tätigkeit als Coach für Studenten vereint die Kompetenzen, einerseits komplexe Sachverhalte schnell zu durchdringen und zugleich Lösungen für Schwierigkeiten bei wissenschaftlichen Arbeiten zu entwickeln. Diese Konzepte verständlich zu kommunizieren, ist zweites Kernstück meiner Coaching-Arbeit. Zu meinen besonderen Stärken gehören eine schnelle Auffassungsgabe und analytisches Denkvermögen, die auch unter Zeitdruck nicht leiden.



Meine vielfältigen Kompetenzen bei Projekten in Ihrem Unternehmen weiterentwickeln zu können, sehe ich als motivierende Herausforderung.


Da ich aktuell selbstständig bin, steht einer zeitnahen Arbeitsaufnahme nichts entgegen. Aufgrund meiner Qualifikation und Kenntnisse liegen meine Gehaltsvorstellungen bei 45.000 Euro im Jahr.



Auf ein Kennenlernen freue ich mich, denn ich überzeuge ich Sie sehr gerne persönlich von meiner Qualifikation. Für ein Gespräch stehe ich Ihnen jederzeit zur Verfügung.


\newpage

Auf ein Kennenlernen freue ich mich, denn ich überzeuge ich Sie sehr gerne persönlich von meiner Qualifikation. Für ein Gespräch stehe ich Ihnen jederzeit zur Verfügung.


  
%Yours sincerely,\\[2em] % if the opening is "Dear Mr(s) Doe,"
Mit freundlichen Grüßen,\\[3em] % if the opening is "Dear Sir or Madam,"
%
%\includegraphics[scale=0.75]{signature_blue}\\
{\bfseries \firstname~\familyname}\\
%
\vfill%
{\slshape \bfseries Bewerbungsunterlagen}\\
 {\slshape Lebenslauf\\
 Arbeitszeugnisse\\
 Diplomzeugnis{}}
\end{document}
